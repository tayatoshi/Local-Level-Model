\documentclass[uplatex]{jsarticle}
\usepackage{amsmath}
%-------------------------------------------------------------------
\begin{document}
\title{ローカルレベルモデル}
\author{田柳俊和}
\maketitle
%-------------------------------------------------------------------
\section{モデル}
\begin{align*}
  y_{t}&=\alpha_{t}+\varepsilon_{t} &\varepsilon_{t} \sim N(0,\sigma_{\varepsilon}^{2})\\
  \alpha_{t}&=\alpha_{t-1}+\eta_{t} &\eta_{t} \sim N(0,\sigma_{\eta}^{2})\\
									   & &\alpha_{1} \sim N(a_{1},P_{1})
\end{align*}
\section{カルマンフィルタ}
\subsection{フィルタリング}
$Y_{t}=\{y_{1} \cdots y_{t}\}$とすると\\
一期先状態予測とその分散
\begin{align*}
  a_{t}&=E(\alpha_{t}|Y_{t-1})\\
  P_{t}&=E(\alpha_{t}|Y_{t-1})
\end{align*}

フィルタ化推定量とその分散
\begin{align*}
  a_{t|t}&=E(\alpha_{t}|Y_{t})\\
  P_{t|t}&=E(\alpha_{t}|Y_{t})
\end{align*}

一期先観測値予測
\begin{align*}
  E(y_{t}|Y_{t-1})&=E(\alpha_{t}+\varepsilon_{t}|Y_{t-1})\\
				  &=E(\alpha_{t}|Y_{t-1})\\
				  &=a_{t}
\end{align*}
一期先観測値予測誤差(イノベーション)とその分散
\begin{align*}
  \upsilon_{t}&=y_{t}- E(y_{t}|Y_{t-1})\\
			  &=y_{t}-a_{t}\\
  F_{t}&=Var(\upsilon_{t}|Y_{t-1})
\end{align*}
%-------------------------------------------------------------------
\newpage
最初からみていくと
\begin{align*}
  a_{1}&=E(\alpha_{1})\\
  P_{1}&=Var(\alpha_{1})
\end{align*}
は既知である.したがって
\begin{align*}
  E(\upsilon_{1})&=E(y_{1}-a_{1})\\
				 &=a_{1}-0-a_{1}\\
				 &=0,\\
  F_{1}&=Var(\upsilon_{1})\\
	   &=Var(y_{1}-a_{1})\\
       &=Var(\alpha_{1}+\varepsilon_{1}-a_{1})\\
       &=E[(E(\alpha_{1}+\varepsilon_{1})-(\alpha_{1}+\varepsilon_{1}))^{2}]\\
       &=E[a_{1}^{2}-2a_{1}(\alpha_{1}+\varepsilon_{1})+\alpha_{1}^{2}+2\alpha_{1}\varepsilon_{1}+\varepsilon_{1}^{2}]\\
       &=E[a_{1}^{2}-2a_{1}\alpha_{1}+\alpha_{1}^{2}+\varepsilon_{1}^{2}]\\
       &=E[(a_{1}-\alpha_{1})^{2}]+E[(0-\varepsilon_{1})^{2}]\\
       &=Var(\alpha_{1})+Var(\varepsilon_{1})\\
       &=P_{1}+\sigma_{\varepsilon}^{2},\\
  Cov(\alpha_{1},\upsilon_{1})&=Cov(\alpha_{1},\alpha_{1}+\varepsilon_{1}-a_{1})\\
  &=E[\alpha_{1}-a_{1}]E[(\alpha_{1}+\varepsilon_{1}-a_{1})-E[\alpha_{1}+\varepsilon_{1}-a_{1}]]\\
  &=E[\alpha_{1}-a_{1}]E[\alpha_{1}+\varepsilon_{1}-a_{1}]\\
  &=E[\alpha_{1}-a_{1}]E[\alpha_{1}-a_{1}]\\
  &=Var(\alpha_{1})\\
  &=P_{1}
\end{align*}
$\upsilon_{1}=y_{1}-a_{1}$は観測値$y_{1}$から定数$a_{1}$を差し引いたものなので,$\upsilon_{1}$と$y_{1}$は一対一対応する.\\
したがって$E(\cdot|y_{1})=E(\cdot|\mu_{1})$となるので,
\begin{align*}
  a_{1|1}&=E(\alpha_{1}|\upsilon_{1})=E(\alpha_{1})+Cov(\alpha_{1},\upsilon_{1})Var(\upsilon_{1})^{-1}(\upsilon_{1}-E(\upsilon_{1}))\footnotemark[1]\\
		 &=E(\alpha_{1})+Cov(\alpha_{1},\upsilon_{1})Var(\upsilon_{1})^{-1}\upsilon_{1}\\
		 &=a_{1}+P_{1}F_{1}^{-1}\upsilon{1}\\
		 &=a_{1}+K_{1}\upsilon_{1}\\
  P(1|1)&=Var(\alpha_{1}|\upsilon_{1})=Var(\alpha_{1})-Cov(\alpha_{1},\upsilon_{1})Var(\upsilon_{1})^{-1}Cov(\upsilon_{1},\alpha_{1})\\
		&=Var(\alpha_{1}|\upsilon_{1})=Var(\alpha_{1})-Cov(\alpha_{1},\upsilon_{1})^{2}Var(\upsilon_{1})^{-1}\\
		&=P_{1}-P_{1}^{2}F_{1}^{-1}\\
		&=P_{1}(1-K_{1})=P_{1}L_{1}
\end{align*}
\footnotetext[1]{多変量正規分布の条件付き期待値と分散}
ここで,$K_{1}=P_{1}/F{1},\ L_{1}=1-K_{1}$とおく.\\
$a_{2},P_{2}$は
\begin{align*}
  a_{2}&=E(\alpha_{2}|y_{1})=E(\alpha_{1}+eta_{1}|y_{1})=a_{1|1}\\
  P_{2}&=P(\alpha_{2}|y_{1})=Var(\alpha_{1}+\eta_{1}|y_{1})=Var(\alpha_{1}|y_{1})+2Cov(\alpha_{1},\eta_{1}|y_{1})+Var(\eta_{1}|y_{1})=P_{1|1}+\sigma_{\eta}^{2}
\end{align*}
\subsection{平滑化}
\subsection{予測}
\subsection{欠損値}
%-------------------------------------------------------------------
\end{document}
